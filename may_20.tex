\documentclass[12pt,letterpaper]{hmcpset}
\usepackage[margin=1in]{geometry} 
\usepackage{graphicx}
\usepackage{amsmath}

% info for header block in upper right hand corner
\name{ }
\class{Math 60}
\assignment{HW 4}
\duedate{Friday, May 20, 2016}

\newcommand{\pn}[1]{\left( #1 \right)}
\newcommand{\abs}[1]{\left| #1 \right|}
\newcommand{\bk}[1]{\left[ #1 \right]}

\renewcommand{\labelenumi}{{(\alph{enumi})}}

\begin{document}

\problemlist{3.1.\{3, 10, 17\}, 3.2.\{1, 4, 13\}, 3.3.\{9, 18, 20\}}

\begin{problem}[3.1.3]
    In Exercises 1-6, sketch the images of the following paths, using
    arrows to indicate the direction in which the parameter increases:
    \[
        \begin{cases}
            x=t\cos t\\
            y=t\sin t
        \end{cases}
        ,\quad-6\pi\leq t\leq6\pi
    \]
\end{problem}
\begin{solution}
    \vfill
\end{solution}
\newpage

\begin{problem}[3.1.10]
    Calculate the velocity, speed, and acceleration of the paths given
    in Exercises 7-10.
    \[
        \textbf{x}(t)=(e^t,e^{2t},2e^t)
    \]
\end{problem}
\begin{solution}
    \vfill
\end{solution}
\newpage

\begin{problem}[3.1.17]
    In Exercises 15-18, find an equation for the line tangent to the
    given path at the indicated value for the parameter.
    \[
        \textbf{x}(t)=(t^2,t^3,t^5),t=2
    \]
\end{problem}
\begin{solution}
    \vfill
\end{solution}
\newpage

\begin{problem}[3.2.1]
    Calculate the length of each of the paths given in Exercises 1-6.
    \[
        \textbf{x}(t)=(2t+1,7-3t),-1\leq t\leq2
    \]
\end{problem}
\begin{solution}
    \vfill
\end{solution}
\newpage

\begin{problem}[3.2.4]
    Calculate the length of each of the paths given in Exercises 1-6.
    \[
        \textbf{x}(t)=7\textbf{i}+t\textbf{j}+t^2\textbf{k},1\leq t\leq3
    \]
\end{problem}
\begin{solution}
    \vfill
\end{solution}
\newpage

\begin{problem}[3.2.13]
    This problem concerns the path $\textbf{x}=|t-1|\textbf{i}
    +|t|\textbf{j},-2\leq t\leq2$.
    \begin{enumerate}
        \item Sketch this path.
        \item The path fails to be of class $C^1$ but is piecewise $C^1$.
            Explain.
        \item Calculate the length of the path.
    \end{enumerate}
\end{problem}
\begin{solution}
    \vfill
\end{solution}
\newpage

\begin{problem}[3.3.9]
    In Exercises 7-12, sketch the given vector field on $\mathbb{R}^3$.
    \textit{Note:} describe in addition to sketch.
    \[
        \textbf{F}=(0,z,-y)
    \]
\end{problem}
\begin{solution}
    \vfill
\end{solution}
\newpage

\begin{problem}[3.3.18]
    In Exercises 17-19, verify that the path given is a flow line of
    the indicated vector field. Justify the result geometrically with
    an appropriate sketch.
    \[
        \textbf{x}(t)=(\sin t,\cos t,2t),\textbf{F}=(y,-x,2)
    \]
\end{problem}
\begin{solution}
    \vfill
\end{solution}
\newpage


\begin{problem}[3.3.20]
    In Exercises 20-22, calculate the flow line \textbf{x}(t) of the
    given vector field \textbf{F} that passes through the indicated
    point at the specified value of $t$.
    \[
        \textbf{F}(x,y)=-x\textbf{i}+y\textbf{j};\quad\textbf{x}(0)=(2,1)
    \]
\end{problem}
\begin{solution}
    \vfill
\end{solution}
\end{document}
