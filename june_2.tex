\documentclass[12pt,letterpaper]{hmcpset}
\usepackage[margin=1in]{geometry}
\usepackage{graphicx}
\usepackage{enumitem}

% info for header block in upper right hand corner
\name{ }
\class{Math 60}
\assignment{HW 12}
\duedate{Thursday, June 2, 2016}

\newcommand{\s}[1]{\sqrt{#1}}
\newcommand{\f}[2]{\frac{#1}{#2}}
\newcommand{\p}[2]{\frac{\partial#1}{\partial#2}}
\newcommand{\bb}[1]{\mathbb{#1}}
\newcommand{\pn}[1]{\left(#1\right)}
\newcommand{\abs}[1]{\left|#1\right|}
\newcommand{\bk}[1]{\left[#1\right]}
\newcommand{\set}[1]{\left\{#1\right\}}
\renewcommand{\bf}[1]{\mathbf{#1}}
\renewcommand{\labelenumi}{{(\alph{enumi})}}

\begin{document}

\problemlist{7.2.\{6, 14, 22, 28\}, 7.3.\{4, 6\}, 7.4.\{T/F pg.522, 6, 10, 18\}}

\begin{problem}[Colley 7.2.6]
    Find $\iint_S(x^2+y^2)~dS$, where $S$ is the lateral surface of
    the cylinder of radius $a$ and height $h$ whose axis is the
    $z$-axis.
\end{problem}
\begin{solution}
    \vfill
\end{solution}
\newpage

\begin{problem}[Colley 7.2.14]
    In Exercises 10-18, let $S$ denote the closed cylinder with bottom
    given by $z=0$, top given by $z=4$, and lateral surface given by
    the equation $x^2+y^2=9$. Orient $S$ with outward
    normals. Determine the indicated scalar and vector surface
    integrals.
    \[
        \iint_S(x~\bf{i}+y~\bf{j})\cdot d\bf{S}
    \]
\end{problem}
\begin{solution}
    \vfill
\end{solution}
\newpage

\begin{problem}[Colley 7.2.22]
    In Exercises 19-22, find the flux of the given vector field
    $\bf{F}$ across the upper hemisphere
    $x^2+y^2+z^2=a^2,~z\geq0$. Orient the hemisphere with an
    upward-pointing normal.
    \[
        \bf{F}=x^2~\bf{i}+xy~\bf{j}+xz~\bf{k}
    \]
\end{problem}
\begin{solution}
    \vfill
\end{solution}
\newpage

\begin{problem}[Colley 7.2.28]
    The glass dome of a futuristic greenhouse is shaped like the
    surface $z=8-2x^2-2y^2$. The greenhouse has a flat dirt floor at
    $z=0$. Suppose that the temperature $T$, at points in and around
    the greenhouse, varies as
    \[
        T(x,y,z)=x^2+y^2+3(z-2)^2.
    \]
    Then the temperature gives rise to a \textbf{heat flux density
    field} $\bf{H}$ given by $\bf{H}=-k\nabla T$. (Here $k$ is a
    positive constant that depends on the insulating properties of the
    particular medium.) Find the total heat flux outward across the
    dome and the surface of the ground if $k=1$ on the glass and $k=3$
    on the ground.
\end{problem}
\begin{solution}
    \vfill
\end{solution}
\newpage

\begin{problem}[Colley 7.3.4]
    In Exercises 1-4, verify Stokes's theorem for the given surface
    and vector field.
    \begin{align*}
        S\text{ is defined by }&x^2+y^2+z^2=4,~z\leq0\text{, oriented
        by downward normal;}\\
        \bf{F}&=(2y-z)~\bf{i}+(x+y^2-z)~\bf{j}+(4y-3x)~\bf{k}
    \end{align*}
\end{problem}
\begin{solution}
    \vfill
\end{solution}
\newpage

\begin{problem}[Colley 7.3.6]
    In Exercises 6-9, verify Gauss's theorem for the given
    three-dimensional region $D$ and vector field $\bf{F}$.
    \begin{align*}
        \bf{F}&=x~\bf{i}+y~\bf{j}+z~\bf{k},\\
        D&=\set{(x,y,z)~|~0\leq z\leq9-x^2-y^2}
    \end{align*}
\end{problem}
\begin{solution}
    \vfill
\end{solution}
\newpage

\begin{problem}[Colley True/False page 522]
    \begin{enumerate}[label=\textbf{\arabic*.}]
        \item The function $\bf{X}:\bb{R}^2\to\bb{R}^3$ given by
            $\bf{X}(s,t)=(2s+3t+1,4s-t,s+2t-7)$ parametrizes the plane
            $9x-y-14z=107$.
        \item The function $\bf{X}:\bb{R}^2\to\bb{R}^3$ given by
            $\bf{X}(s,t)=(s^2+3t-1,s^2+3,-2s^2+t)$ parametrizes the
            plane $x-7y-3z+22=0$.
        \item The function
            $\bf{X}:(-\infty,\infty)\times(-\f{\pi}{2},\f{\pi}{2})\to\bb{R}^3$
            given by $\bf{X}(s,t)=(s^3+3\tan t-1,s^3+3,-2s^3+\tan t)$
            parametrizes the plane $x-7y-3z+22=0$.
        \item The surface $\bf{X}(s,t)=(s^2t,st^2,st)$ is smooth.
    \end{enumerate}
\end{problem}
\begin{solution}
    \vfill
\end{solution}
\newpage

\begin{problem}[Colley 7.4.6]
    Exercises 6-10 concern these notions of temperature, heat, and
    heat flux density.
    \begin{align*}
        \text{Use Gauss's theorem to }&\text{derive the
        \textbf{heat equation},}\\
        \sigma\rho\p{T}{t}&=k\nabla^2T.
    \end{align*}
\end{problem}
\begin{solution}
    \vfill
\end{solution}
\newpage

\begin{problem}[Colley 7.4.10]
    Exercises 6-10 concern these notions of temperature, heat, and
    heat flux density.\\\\
    Consider the three-dimensional heat equation
    \begin{equation}\tag{30}
        \nabla^2u=\p{u}{t}
    \end{equation}
    for functions $u(x,y,z,t)$. (Here $\nabla^2u$ denotes the
    Laplacian $\partial^2u/\partial x^2+\partial^2 u/\partial
    y^2+\partial^2u/\partial z^2$.) In this exercise, show that any
    solution $T(x,y,z,t)$ to the heat equation is unique in the
    following sense: Let $D$ be a bounded solid region in $\bb{R}^3$
    and suppose that the functions $\alpha(x,y,z)$ and $\phi(x,y,z,t)$
    are given. Then there exists a unique solution $T(x,y,z,t)$ to
    equation (30) that satisfies the conditions
    \begin{align*}
        T(x,y,z,0)&=\alpha(x,y,z)\quad\text{for }(x,y,z)\in D,\\
        \text{and}\hspace{5mm}\tag{31}\\
        T(x,y,z,t)&=\phi(x,y,z,t)\quad\text{for }(x,y,z)\in \partial D
        \text{ and }t\geq0.
    \end{align*}
    To establish uniqueness, let $T_1$ and $T_2$ be two solutions to
    equation (30) satisfying the conditions in (31) and set
    $w=T_1-T_2$.
    \begin{enumerate}
        \item Show that $w$ must also satisfy equation (30), plus the
            conditions that
            \begin{align*}
                w(x,y,z,0)&=0\quad\text{ for all }(x,y,z)\in D,\\
                \text{and}\hspace{5mm}\\
                w(x,y,z,t)&=0\quad\text{ for all }(x,y,z)\in\partial D
                \text{ and }t\geq0.
            \end{align*}
        \item For $t\geq0$, define the ``energy function''
            \[
                E(t)=\f{1}{2}\iiint_D[w(x,y,z,t)]^2~dV.
            \]
            Use Green's first formula in Theorem 4.1 to show that
            $E'(t)\leq0$ (i.e., that $E$ does not increase with time).
        \item Show that $E(t)=0$ for all $t\geq0$. (Hint: Show that
            $E(0)=0$ and use part (b).)
        \item Show that $w(x,y,z,t)=0$ for all $t\geq0$ and $(x,
            y,z)\in D$, and thereby conclude the uniqueness of solutions
            to equation (30) that satisfy the conditions in (31).
    \end{enumerate}
    \textit{Hint:} think about what the field is doing.
\end{problem}
\newpage
\begin{solution}
    \null\vfill
\end{solution}
\newpage

\begin{problem}[Colley 7.4.18]
    Suppose that $\bf{J}=\sigma\bf{E}$. (This is a version of Ohm's
    law that obtains in some electric conductors—--here $\sigma$ is a
    positive constant known as the \textbf{conductivity}.) If
    $\rho=0$, show that $\bf{E}$ and $\bf{B}$ satisfy the so-called
    \textbf{telegrapher's equation},
    \[
        \nabla^2\bf{F}=\mu_0\sigma\p{\bf{F}}{t}
        +\mu_0\epsilon_0\p{^2\bf{F}}{t^2}.
    \]
\end{problem}
\begin{solution}
    \vfill
\end{solution}
\end{document}
