\documentclass[12pt,letterpaper]{hmcpset}
\usepackage[margin=1in]{geometry} 
\usepackage{graphicx}
\usepackage{amsmath}

% info for header block in upper right hand corner
\name{ }
\class{Math 60}
\assignment{HW 3}
\duedate{Thursday, May 19, 2016}

\newcommand{\pn}[1]{\left( #1 \right)}
\newcommand{\abs}[1]{\left| #1 \right|}
\newcommand{\bk}[1]{\left[ #1 \right]}

\newcommand{\pf}[2]{\frac{\partial#1}{\partial#2}}

\renewcommand{\labelenumi}{{(\alph{enumi})}}

\begin{document}

\problemlist{2.5.\{6, 14, 24, 36\}, 2.6.\{6, 12, 18\}}

\begin{problem}[2.5.6]
    A rectangular stick of butter is placed in the microwave oven to
    melt. When the butter's length is 6 in and its square cross
    section measures 1.5 in on a side, its length is decreasing at a
    rate of 0.25 in/min and its cross-sectional edge is decreasing at
    a rate of 0.125 in/min. How fast is the butter melting (i.e., at
    what rate is the solid volume of butter turning to liquid) at that
    instant?
\end{problem}
\begin{solution}
    \vfill
\end{solution}
\newpage

\begin{problem}[2.5.14]
    Suppose that $z=f(x+y,x-y)$ has continuous partial derivatives
    with respect to $u=x+y$ and $v=x-y$. Show that
    \[
        \pf{z}{x}\pf{z}{y}=\pn{\pf{z}{u}}^2-\pn{\pf{z}{v}}^2.
    \]
\end{problem}
\begin{solution}
    \vfill
\end{solution}
\newpage

\begin{problem}[2.5.24]
    In Exercises 19-27, calculate $D(\textbf{f}\circ\textbf{g})$ in
    two ways: (a) by first evaluating $\textbf{f}\circ\textbf{g}$ and
    (b) by using the chain rule and the derivative matrices
    $D\textbf{f}$ and $D\textbf{g}$.\\\\
    $\textbf{f}(x,y,z)=(x^2y+y^2z,xyz,e^z),\\
    \textbf{g}(t)=(t-2,3t+7,t^3)$
\end{problem}
\begin{solution}
    \vfill
\end{solution}
\newpage

\begin{problem}[2.5.36]
    Suppose that you are given an equation of the form
    \[
        F(x,y,z)=0,
    \]
    for example, something like $x^3z+y\cos z+(\sin y)/z=0$. Then we
    may consider $z$ to be defined implicitly as a function $z(x,y)$.
    \begin{enumerate}
        \item Use the chain rule to show that if $F$ and $z(x,y)$ are
            both assumed to be differentiable, then
            \[
                \pf{z}{x}=-\frac{F_x(x,y,z)}{F_z(x,y,z)},\quad
                \pf{z}{y}=-\frac{F_y(x,y,z)}{F_z(x,y,z)}.
            \]
        \item Use part (a) to find $\partial z/\partial x$ and
            $\partial z/\partial y$ where $z$ is given by the equation
            $xyz=2$. Check your result by explicitly solving for $z$ and
            then calculating the partial derivatives.
    \end{enumerate}
\end{problem}
\begin{solution}
    \vfill
\end{solution}
\newpage

\begin{problem}[2.6.6]
    In Exercises 2-8, calculate the directional derivative of the
    given function $f$ at the point \textbf{a} in the direction
    parallel to the vector \textbf{u}.\\\\
    $\displaystyle f(x,y,z)=xyz,\textbf{a}=(-1,0,2),
    \textbf{u}=\frac{2\textbf{k}-\textbf{i}}{\sqrt{5}}$
\end{problem}
\begin{solution}
    \vfill
\end{solution}
\newpage

\begin{problem}[2.6.12]
    A ladybug (who is very sensitive to temperature) is crawling on
    graph paper. She is at the point $(3,7)$ and notices that if she
    moves in the $\textbf{i}$-direction, the temperature
    \textit{increases} at a rate of 3 deg/cm. If she moves in the
    $\textbf{j}$-direction, she finds that her temperature
    \textit{decreases} at a rate of 2 deg/cm. In what direction should
    the ladybug move if
    \begin{enumerate}
        \item she wants to warm up most rapidly?
        \item she wants to cool off most rapidly?
        \item she desires her temperature \textit{not} to change?
    \end{enumerate}
\end{problem}
\begin{solution}
    \vfill
\end{solution}
\newpage

\begin{problem}[2.6.18]
    In Exercises 16-19, find an equation for the tangent plane to the
    surface given by the equation at the indicated point
    $(x_0,y_0,z_0)$.\\\\
    $2xz+yz-x^2y+10=0,(x_0,y_0,z_0)=(1,-5,5)$
\end{problem}
\begin{solution}
    \vfill
\end{solution}
\end{document}
