\documentclass[12pt,letterpaper]{hmcpset}
\usepackage[margin=1in]{geometry} 
\usepackage{graphicx}
\usepackage{amsmath}

% info for header block in upper right hand corner
\name{ }
\class{Math 60}
\assignment{HW 2}
\duedate{Wednesday, May 18, 2016}

\newcommand{\pn}[1]{\left( #1 \right)}
\newcommand{\abs}[1]{\left| #1 \right|}
\newcommand{\bk}[1]{\left[ #1 \right]}

\renewcommand{\labelenumi}{{(\alph{enumi})}}

\begin{document}

\problemlist{2.3.\{24, 33, 40, 42\}, 2.4.\{5, 17, 23, 29a\}}

\begin{problem}[2.3.24]
    Find the gradient $\nabla$ $f(\textbf{a})$, where $f$ and
    \textbf{a} are given in Exercises 18-25.\\\\
    $f(x,y,z)=\cos z\ln(x+y^2),\quad \textbf{a}=(e,0,\pi/4)$
\end{problem}
\begin{solution}
    \vfill
\end{solution}
\newpage

\begin{problem}[2.3.33]
    In Exercises 26-33, find the matrix D\textbf{f(a)} of partial
    derivatives, where \textbf{f} and \textbf{a} are as indicated.\\\\
    $\textbf{f}(s,t)=(s^2,st,t^2),\quad \textbf{a}=(−1,1)$
\end{problem}
\begin{solution}
    \vfill
\end{solution}
\newpage

\begin{problem}[2.3.40]
    Find equations for the planes tangent to $z=x^2-6x+y^3$ that are parallel to
    the plane $4x-12y+z=7$.
\end{problem}
\begin{solution}
    \vfill
\end{solution}
\newpage

\begin{problem}[2.3.42]
    Suppose that you have the following information concerning a
    differentiable function $f$:
    $$f(2,3)=12,\quad f(1.98,3)=12.1,\quad f(2,3.01)=12.2.$$
    \begin{enumerate}
        \item Give an approximate equation for the plane tangent to the graph of
            $f$ at (2,3,12).
        \item Use the result of part (a) to estimate $f(1.98, 2.98)$.
    \end{enumerate}
\end{problem}
\begin{solution}
    \vfill
\end{solution}
\newpage

\begin{problem}[2.4.5]
    Verify the product and quotient rules (Proposition 4.2) for the
    pairs of functions given in Exercises 5-8.\\\\
    $\displaystyle f(x,y)=x^2y+y^3,\quad g(x,y)=\frac{x}{y}$
\end{problem}
\begin{solution}
    \vfill
\end{solution}
\newpage

\begin{problem}[2.4.17]
    For the functions given in Exercises 9-17 determine all
    second-order partial derivatives (including mixed partials).\\\\
    $f(x,y)=x^2e^y+e^{2z}$
\end{problem}
\begin{solution}
    \vfill
\end{solution}
\newpage

\begin{problem}[2.4.23]
    Let $f(x,y)=ye^{3x}$. Give general formulas for $\partial^n f/\partial x^n$
    and $\partial^n f/\partial y^n$, where $n\geq 2$.
\end{problem}
\begin{solution}
    \vfill
\end{solution}
\newpage

\begin{problem}[2.2.29a]
    The three-dimensional heat equation is the partial differential equation
    $$k\pn{\frac{\partial^2 T}{\partial x^2}+\frac{\partial^2 T}{\partial y^2}
    +\frac{\partial^2 T}{\partial z^2}}=\frac{\partial T}{\partial t}$$
    where $k$ is a positive constant. It models the temperature
    $T(x,y,z,t)$ at the point $(x,y,z)$ and time $t$ of a body in space.
    \begin{enumerate}
        \item We examine a simplified version of the heat
            equation. Consider a straight wire ``coordinatized'' by $x$. Then
            the temperature $T(x,t)$ at time $t$ and position $x$ along
            the wire is modeled by the one-dimensional heat equation
            $$k\frac{\partial^2 T}{\partial x^2}
            =\frac{\partial T}{\partial t}.$$
            Show that the function $T(x,t)=e^{-kt}\cos x$ satisfies this
            equation. Note that if $t$ is held constant at value $t_0$, then
            $T(x,t_0)$ shows how the temperature varies along the wire at time
            $t_0$. Graph the curves $z=T(x,t_0)$ for $t_0=0,1,10$, and use them to
            understand the graph of the surface $z=T(x,t)$ for $t\geq0$. Explain
            what happens to the temperature of the wire after a long period of
            time.
    \end{enumerate}
\end{problem}
\begin{solution}
    \vfill
\end{solution}
\end{document}

